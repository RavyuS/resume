%%%%%%%%%%%%%%%%%
% This is an sample CV template created using altacv.cls
% (v1.3, 10 May 2020) written by LianTze Lim (liantze@gmail.com). Now compiles with pdfLaTeX, XeLaTeX and LuaLaTeX.
%
%% It may be distributed and/or modified under the
%% conditions of the LaTeX Project Public License, either version 1.3
%% of this license or (at your option) any later version.
%% The latest version of this license is in
%%    http://www.latex-project.org/lppl.txt
%% and version 1.3 or later is part of all distributions of LaTeX
%% version 2003/12/01 or later.
%%%%%%%%%%%%%%%%

%% If you are using \orcid or academicons
%% icons, make sure you have the academicons
%% option here, and compile with XeLaTeX
%% or LuaLaTeX.
% \documentclass[10pt,a4paper,academicons]{altacv}

%% Use the "normalphoto" option if you want a normal photo instead of cropped to a circle
% \documentclass[10pt,a4paper,normalphoto]{altacv}

\documentclass[10pt,a4paper,ragged2e,withhyper]{altacv}
%% AltaCV uses the fontawesome5 and academicons fonts
%% and packages.
%% See http://texdoc.net/pkg/fontawesome5 and http://texdoc.net/pkg/academicons for full list of symbols. You MUST compile with XeLaTeX or LuaLaTeX if you want to use academicons.

% Change the page layout if you need to
\geometry{left=1.25cm,right=1.25cm,top=1.5cm,bottom=1.5cm,columnsep=1.2cm}

% The paracol package lets you typeset columns of text in parallel
\usepackage{paracol}

% Change the font if you want to, depending on whether
% you're using pdflatex or xelatex/lualatex
\ifxetexorluatex
  % If using xelatex or lualatex:
  \setmainfont{Roboto Slab}
  \setsansfont{Lato}
  \renewcommand{\familydefault}{\sfdefault}
\else
  % If using pdflatex:
  \usepackage[rm]{roboto}
  \usepackage[defaultsans]{lato}
  % \usepackage{sourcesanspro}
  \renewcommand{\familydefault}{\sfdefault}
\fi

% Change the colours if you want to
\definecolor{SlateGrey}{HTML}{2E2E2E}
\definecolor{LightGrey}{HTML}{666666}
\definecolor{DarkPastelRed}{HTML}{450808}
\definecolor{PastelRed}{HTML}{8F0D0D}
\definecolor{GoldenEarth}{HTML}{E7D192}
\colorlet{name}{black}
\colorlet{tagline}{PastelRed}
\colorlet{heading}{DarkPastelRed}
\colorlet{headingrule}{GoldenEarth}
\colorlet{subheading}{PastelRed}
\colorlet{accent}{PastelRed}
\colorlet{emphasis}{SlateGrey}
\colorlet{body}{LightGrey}

% Change some fonts, if necessary
\renewcommand{\namefont}{\Huge\rmfamily\bfseries}
\renewcommand{\personalinfofont}{\footnotesize}
\renewcommand{\cvsectionfont}{\LARGE\rmfamily\bfseries}
\renewcommand{\cvsubsectionfont}{\large\bfseries}


% Change the bullets for itemize and rating marker
% for \cvskill if you want to
\renewcommand{\itemmarker}{{\small\textbullet}}
\renewcommand{\ratingmarker}{\faCircle}

%% sample.bib contains your publications
% \addbibresource{sample.bib}

\begin{document}
\name{Ravyu}
\tagline{Undergraduate CS Student}
%% You can add multiple photos on the left or right
% \photoR{2.8cm}{portrait}
% \photoL{2.5cm}{Yacht_High,Suitcase_High}

\personalinfo{%
  % Not all of these are required!
  \email{ravyus2@illinois.edu}
  \phone{217-417-8687}
  % \mailaddress{Åddrésş, Street, 00000 Cóuntry}
  \location{Champaign, IL}\\
  % \homepage{www.homepage.com}
  % \twitter{@twitterhandle}
  \linkedin{Ravyu}
  \github{RavyuS}
  \printinfo{\faUniversity}{GPA (cumulative): 4.0/4.0}
  %% You MUST add the academicons option to \documentclass, then compile with LuaLaTeX or XeLaTeX, if you want to use \orcid or other academicons commands.
  % \orcid{0000-0000-0000-0000}
  %% You can add your own arbtrary detail with
  %% \printinfo{symbol}{detail}[optional hyperlink prefix]
  % \printinfo{\faPaw}{Hey ho!}[https://example.com/]
  %% Or you can declare your own field with
  %% \NewInfoFiled{fieldname}{symbol}[optional hyperlink prefix] and use it:
  % \NewInfoField{gitlab}{\faGitlab}[https://gitlab.com/]
  % \gitlab{your_id}
}

\makecvheader
%% Depending on your tastes, you may want to make fonts of itemize environments slightly smaller
% \AtBeginEnvironment{itemize}{\small}

%% Set the left/right column width ratio to 6:4.
\columnratio{0.6}

% Start a 2-column paracol. Both the left and right columns will automatically
% break across pages if things get too long.
\begin{paracol}{2}
\cvsection{Work Experience}
\cvevent{Software Design Studio Course Assistant}{UIUC CS Department}{Jan 2021 -- May 2021}{Champaign, IL}
\begin{cvitemize}
  \item Taught students on programming fundamentals: Object Oriented Programming (Java/C++), black box testing, build automation tools (Cmake)
  \item Assessed and graded students code on weekly projects, with an emphasis on strong programming practices (modularity, object decomposition, encapsulation, documentation, testing)
  \item Hosted weekly Code Review sessions to facillitate discussion on programming practices with students
\end{cvitemize}

\divider

\cvevent{Backend Developer}{CloudZen Pte., Ltd.,}{Jan 2018 -- Apr 2018}{Singapore}
\begin{cvitemize}
\item Developed backend infrastructure for the Company’s flagship project—Gamecloud. Setup a Leaderboard database (MongoDB + Redis) that is linked via API endpoints defined in Azure Functions
\item Designed network pipeline for latency sensitive applications like game streaming over both cellular and WAN connections. Contributed to the UDP socket programming team
\item Spearheaded virtualization infrastructure setup with focus on performance for gaming. The virtualization infrastructure was built on a KVM-based virtualization platform, with Proxmox used for cluster management

\end{cvitemize}

\cvsection{Leadership}

\cvevent{Lead Programmer}{ACS(I) Robotics Club}{Jan 2016 -- Nov 2017}{Singapore}

\begin{itemize}
  \item Led club as head programmer during competitions (VEX Worlds, RoboCup). Coordinated weekly meetings leading up to the competitions, and guided junior members along the way.
  \item Programmed microcontrollers in various instruction sets (ARM Cortex-M, Atmel AVR) in embedded C and assembly. 
  \item Designed custom circuit boards, experienced in soldering and EAGLE Design.
\end{itemize}

\divider

\cvevent{Air Warfare Officer}{Singapore Armed Forces}{Apr 2018 -- Apr 2020}{Singapore}
\begin{itemize}
\item Trained as Air Traffic Control (ATC) Officer; in charge of recovering air assets in the event of national emergencies.
\item Led ATC operations briefs on a weekly and daily basis. Managed communication between various base squadrons during day-to-day operations.
\item Completed service on April 10th with the rank of Lieutenant.
\end{itemize}

\divider

\cvevent{President}{Singapore Students Association}{Mar 2021 -- Mar 2022}{Champaign, IL}
\begin{cvitemize}
\item Led a committee of six to manage an organization of approximately sixty members
\item Spearheaded several initiatives including partnerships with other cultural societies and sponsorships from the Singapore Government for community-wide events
\item Planned and executed a several-month long orientation program for newly-matriculated Singaporeans, ending off with a overnight guided trip to Chicago.
\end{cvitemize}



\cvsection{Projects}

\cvevent{Air Traffic Control Currency Tracker}{Republic of Singapore Air Force}{Apr 2019 -- Apr 2020}{Singapore}
\begin{itemize}
\item Revamped Air Traffic Controller eligibility tracking system in Squadron to ensure controllers’ skills remain current, thus preventing ineligible controllers from controlling. 
\item The project made use of the Telegram Bot framework for the frontend and Google Apps Script (JavaScript) for the backend, with a focus on high-availability and security to meet operations-sensitive demands.
\item Presented monthly progress presentations to base commanders, showcasing product demo builds.
\end{itemize}

\divider

\cvevent{Open Source Initiative}{ACS(I) Robotics Club}{Nov 2016 -- Apr 2017}{Singapore}
Led an initiative to switch to open source robotics platforms for use in Competitions. 
\medskip
\begin{itemize}
  \item Conducted workshops on how to read instruction set manuals for Arduino (AVR) and datasheets for sensors/electronics, with an emphasis on appreciation for open source culture.
  \item Taught basic computer architecture concepts (ALUs, registers, IO) to junior members.
  \item Setup and maintained communication channels between the software team, hardware team and part suppliers.
\end{itemize}

\divider

\cvevent{Self-Hosted cloud server}{}{Ongoing}{}
Converted my desktop to a cloud server for personal use. 
\medskip
\begin{itemize}
  \item Running full LAMP stack with Nextcloud (Cloud storage), Gitea (Git server), SSH, VNC.
  \item Utilized Linux OS specific tools (network management, bash scripting, systemd, software RAID) to setup the hosting platform.
  \item Working on secure remote WOL (Wake on LAN) solution.
\end{itemize}


\medskip

% \cvsection{A Day of My Life}

% % Adapted from @Jake's answer from http://tex.stackexchange.com/a/82729/226
% % \wheelchart{outer radius}{inner radius}{
% % comma-separated list of value/text width/color/detail}
% \wheelchart{1.5cm}{0.5cm}{%
%   6/8em/accent!30/{Sleep,\\beautiful sleep},
%   3/8em/accent!40/Hopeful novelist by night,
%   8/8em/accent!60/Daytime job,
%   2/10em/accent/Sports and relaxation,
%   5/6em/accent!20/Spending time with family
% }

% use ONLY \newpage if you want to force a page break for
% ONLY the current column
% \newpage

% \cvsection{Publications}

% \nocite{*}

% \printbibliography[heading=pubtype,title={\printinfo{\faBook}{Books}},type=book]

% \divider

% \printbibliography[heading=pubtype,title={\printinfo{\faFile*[regular]}{Journal Articles}},type=article]

% \divider

% \printbibliography[heading=pubtype,title={\printinfo{\faUsers}{Conference Proceedings}},type=inproceedings]

%% Switch to the right column. This will now automatically move to the second
%% page if the content is too long.
\switchcolumn

\cvsection{Education}
\cvevent{Bachelors in Science,\ Computer Science}{University of Illinois at Urbana-Champaign }{May 2023 GPA: 4.0/4.0}{}

\divider

% \cvsection{My Life Philosophy}

% \begin{quote}
% ``Something smart or heartfelt, preferably in one sentence.''
% \end{quote}

\cvsection{Notable Awards}

\cvachievement{\faTrophy}{VEX Worlds}{2017 Division Finalist \newline 2016 Division Quarter-Finalist}

\divider

\cvachievement{\faTrophy}{VEX Skills Programming Qualifiers 2016}{2nd in World rankings (out of 4000 teams)}
% \divider

% \cvachievement{\faHeart}{My little desktop}{By my side since 2011 :)}


% \cvachievement{\faTrophy}{DSTA Code_EXP 2020 Finalist}{Robotics \+ Computer Vision hackathon, University category}

% \cvsection{Strengths}

% \cvtag{Hard-working}
% \cvtag{Eye for detail}\\
% \cvtag{Motivator \& Leader}

% \divider\smallskip

% \cvtag{C++}
% \cvtag{Embedded Systems}\\
% \cvtag{Statistical Analysis}

\cvsection{Technical Skills}

\cvskill{C/C++}{4}
\smallskip
\cvskill{Java}{3}

\divider

\cvskill{LAMP stack + AWS}{3}
\smallskip
\cvskill{React-Native}{2}

\divider

\cvskill{Embedded Systems}{2}
\smallskip
\cvskill{Autonomous Systems}{1}

\divider
\cvskill{TensorFlow/PyTorch}{1}

%% Yeah I didn't spend too much time making all the
%% spacing consistent... sorry. Use \smallskip, \medskip,
%% \bigskip, \vpsace etc to make ajustments.
\medskip

\cvsection[]{Coursework}
\smallskip

\cvtag{Algorithms and Models of Computation}
\cvtag{Data Structures (C++)}
\cvtag{Computer Architecture}
\cvtag{Probability and Statistics}\\
\cvtag{Applied Linear Algebra (Python)}
\cvtag{Discrete Structures}
\cvtag{Programming Studio (Java/C++)}
\cvtag{Calculus I-III}




% \cvevent{IB Diploma}{Anglo-Chinese School (Independent)}{Jan 2016 -- Dec 2017}{Singapore}
% 44 points
% \divider

% \cvsection{Referees}

% % \cvref{name}{email}{mailing address}
% \cvref{Prof.\ Alpha Beta}{Institute}{a.beta@university.edu}
% {Address Line 1\\Address line 2}

% \divider

% \cvref{Prof.\ Gamma Delta}{Institute}{g.delta@university.edu}
% {Address Line 1\\Address line 2}
\newpage

\cvsection[]{Other Achievements}
\cvachievement{\faTrophy}{DSTA Code\_EXP 2020 Finalist}{National mobile app hackathon; University category}
\cvachievement{\faTrophy}{DSTA TIL 2020 Finalist}{Machine Learning \+ Robotics based hackathon; University category}
\cvachievement{\faTrophy}{Singapore VEX Nationals 2016}{Tertiary Division Champions \newline Tournament Excellence Award \newline Programming Skills Champions}
\cvachievement{\faTrophy}{CodeForCorona Hackathon 2020 Finalist}{Open category}
\cvachievement{\faTrophy}{RoboCup Singapore 2017}{4th Placing}
\cvachievement{\faSchool}{School Awards}{Certificate of Distinction for contribution in Robotics--2016, 2017 \newline \smallskip Distinguished Service Award for Exemplary Leadership--2017}


\end{paracol}


\end{document}